\documentclass[a4paper,11pt]{article}
\usepackage{import}
%\usepackage{projectstyle}

%de naam hier is de naam van de file in de sty folder niet de naam van providespackage
\usepackage{sty/1loadrequiredpackagesatveryfirst}
\usepackage{sty/2metadata}
\usepackage{sty/3tocstyle}
\usepackage{sty/4appendixstyle}
\usepackage{sty/5colorstyles}
\usepackage{sty/6sectionsstyle}
\usepackage{sty/7newcommandsfilip}
\usepackage{sty/8glossariesstyle}
\usepackage{sty/9imagesstyle}
\usepackage{sty/10liststyles}
\usepackage{sty/11headerfooterstyle}
\usepackage{sty/12mathstyles}
\usepackage{sty/13customstylesfromothersites}
\usepackage{sty/14stylesforalgoritmsandsourcecode}
\usepackage{sty/15stylefortodonotes}
%\usepackage{sty/}
%\usepackage{sty/}
%\usepackage{sty/}
%\usepackage{sty/}
%\usepackage{sty/}
\usepackage{array}
\usepackage{makeidx}
\usepackage{multirow}


%It prints out all index entries in the left margin of the text. This is quite useful for proofreading a document and verifying the index. For more information, see the Indexing section.

%https://en.wikibooks.org/wiki/LaTeX/Indexing

\usepackage{showidx} %It prints out all index entries in the left margin of the text.

\makeindex

%we have to define a bibliography style in the preamble
\bibliographystyle{plain}

\usepackage{booktabs}


\begin{document}


%\glsaddall
%\import{./}{title.tex}

%\clearpage

\import{./}{title2.tex}

\clearpage

\thispagestyle{empty}

\tableofcontents

\clearpage

%hoofdstuk 4
\setcounter{section}{3}
\newpage
\import{sections/hoofdstuk4sections/}{deel1.tex}
\import{sections/hoofdstuk4sections/}{deel2.tex}
\import{sections/hoofdstuk4sections/}{deel3.tex}
\import{sections/hoofdstuk4sections/}{deel4.tex}
\import{sections/hoofdstuk4sections/}{deel5.tex}

%hoofdstuk 5
%\setcounter{section}{4}
\newpage
\import{sections/hoofdstuk5sections/}{deel1.tex}
\import{sections/hoofdstuk5sections/}{deel2.tex}
\import{sections/hoofdstuk5sections/}{deel3.tex}
\import{sections/hoofdstuk5sections/}{deel4.tex}
\import{sections/hoofdstuk5sections/}{deel5.tex}
\import{sections/hoofdstuk5sections/}{deel6.tex}

%hoofdstuk 6
%\setcounter{section}{5}
\newpage
\import{sections/hoofdstuk6sections/}{deel1.tex}
\import{sections/hoofdstuk6sections/}{deel2.tex}
\import{sections/hoofdstuk6sections/}{deel3.tex}
\import{sections/hoofdstuk6sections/}{deel4.tex}
\import{sections/hoofdstuk6sections/}{deel5.tex}
\import{sections/hoofdstuk6sections/}{deel6.tex}
\import{sections/hoofdstuk6sections/}{deel7.tex}
\import{sections/hoofdstuk6sections/}{deel8.tex}
\import{sections/hoofdstuk6sections/}{deel9.tex}
\import{sections/hoofdstuk6sections/}{deel10.tex}


%hoofdstuk 7
%\setcounter{section}{6}
\newpage
\import{sections/hoofdstuk7sections/}{deel1.tex}
\import{sections/hoofdstuk7sections/}{deel2.tex}
\import{sections/hoofdstuk7sections/}{deel3.tex}
\import{sections/hoofdstuk7sections/}{deel4.tex}


%hoofdstuk 8
%\setcounter{section}{7}
\newpage
\section{Multimedia netwerken}
\import{sections/hoofdstuk8sections/}{deel1.tex}
\import{sections/hoofdstuk8sections/}{deel2.tex}
\import{sections/hoofdstuk8sections/}{deel3.tex}
\import{sections/hoofdstuk8sections/}{deel4.tex}
\import{sections/hoofdstuk8sections/}{deel5.tex}

%hoofdstuk 9
%\setcounter{section}{8}
\newpage
\section{Multimedia netwerken}
\import{sections/hoofdstuk9sections/}{deel1.tex}
\import{sections/hoofdstuk9sections/}{deel2.tex}

\begin{comment}
%toont source code als algoritme
\begin{algorithm}[ht]
%dit zorgt ervoor dat het in je list of algoritms wordt getoond
\caption{See how easy it is to provide algorithms}
\label{myFirstAlgorithm}
\begin{algorithmic}
\REQUIRE $a$
\STATE $b = 0$
\STATE $x \leftarrow 1:10$
\FORALL{x}
    \STATE $b = b+a$
\ENDFOR
\RETURN $b$
\end{algorithmic}
\end{algorithm}

\end{comment}


%Next we are adding the additional lists to the end of the document:

\listofalgorithms
\clearpage
\listoffigures
\clearpage
\listoftables
\clearpage

%\printglossaries
\printglossary[title=Termen,toctitle=Lijst van termen]

\printglossary[type=\acronymtype]


\clearpage
%\import{./}{bibliography.tex}

\end{document}
