\subsection{Algoritmen voor scheduling}

\subsubsection{Criteria bij scheduling voor korte termijn}

\subsubsection{Gebruik van prioriteiten}

De scheduler zal altijd een proces met hogere prioriteit kiezen boven een proces met lagere prioriteit. Er zijn meerdere ready queues om elk niveau prioriteit te weergeven.

\begin{figure}[htp]
    \centering
            \includegraphics[width=4in]{img/gebruikvanprioriteiten.png}
        \caption{Gebruik van prioriteiten voorbeeld}
    \label{fig:Gebruik van prioriteiten voorbeeld}
\end{figure}
 
Hierboven zie hoe wachtrijen van prioriteiten in zijn werk gaan. Er wordt eerst gekeken of er een proces in RQ0 zit, zit er geen in dan gaat men kijken in RQ1, zit er dan geen in, dan kijkt men in RQ2, enz.. Hier bestaat wel kans op starvation! Als oplossing kan het proces zelf zijn prioriteit aanpassen op basis van zijn leeftijd of uitvoeringsgeschiedenis.


\subsubsection{Mogelijke strategieën voor scheduling}

\begin{figure}[htp]
    \centering
            \includegraphics[width=4in]{img/schedulingpolicies.png}
        \caption{Scheduling Policies}
    \label{fig:Scheduling Policies}
\end{figure}

De selectiefunctie bepaald welk proces wordt geselecteerd voor uitvoering. Als het gebaseerd is op uitvoeringsgeschiedenis dan zijn de belangrijkste kwantitatieve kenmerken:
\begin{itemize}
\item W = tijd gespendeerd al wachtend
\item E = tijd gespendeerd aan uitvoering
\item S = totale bedieningstijd die vereist is voor het proces, inclusief E.
\end{itemize}

De beslissingsmodus bepaalt op welke tijdstippen de selectiefunctie wordt uitgevoerd. Hierin bestaan twee categorieën:
\begin{itemize}
\item Niet-preëmptief: eenmaal een proces ‘running’ is, zal het voortdoen totdat het uitgevoerd is of zichzelf blokkeert voor I/O.
\item Preëmptief: de ‘running’ toestand mag onderbroken worden en het proces mag in ‘ready’ toestand gebracht worden door het besturingssysteem. Dit kan optreden wanneer een nieuw proces wordt aangemaakt, of tijdens een interrupt of periodiek op basis van een klokinterrupt.
\end{itemize}

Voorbeeld van proces scheduling:

\begin{figure}[htp]
    \centering
            \includegraphics[width=4in]{img/processcheduling.png}
        \caption{Process Scheduling voorbeeld}
    \label{fig:Process Scheduling voorbeeld}
\end{figure}

\textbf{First come, First served}

Elk proces komt aan in de ready queue bij creatie. Wanneer het proces dat huidig wordt uitgevoerd wordt klaar is, wordt het proces dat het langste in de ready queue is geselecteerd.


\begin{figure}[htp]
    \centering
            \includegraphics[width=4in]{img/firstcomefirstserved.png}
        \caption{Voorbeeld van First come, First served}
    \label{fig:First come, First served voorbeeld}
\end{figure}

Een klein proces kan wel heel lang wachten vooraleer het zijn beurt is op uitvoering. Een ander probleem is dat processorgebonden processen voorrang krijgen op I/O-gebonden processen.

\textbf{Round Robin}

Round Robin gebruikt preëmptieve onderbreking gebaseerd op een klok, dit is ook gekend als time slicing. Elk proces krijgt een slice of time (tijdsperiode) waarin hij mag uitgevoerd worden.


\begin{figure}[htp]
    \centering
            \includegraphics[width=4in]{img/roundrobin.png}
        \caption{Round Robin voorbeeld}
    \label{fig:Round Robin voorbeeld}
\end{figure}

De klokinterrupt wordt altijd met dezelfde regelmaat gegeven. Wanneer deze interrupt gebeurd wordt het ‘running’ proces in de ready queue geplaatst en wordt het eerste proces in de ready queue geselecteerd.

\textbf{Nadeel} van Round Robin is dat processen misschien tijd teveel zouden kunnen hebben. De Slice of Time kan te groot zijn voor hetgeen wat zij moeten doen. Als dit voor alle processen zo is dan komt Round Robin overeen met FCFS. Zeer korte slice of time betekent dat er veel te veel overhead zou zijn, de processor zou meer bezig zijn met het inlezen van processen uit de ready queue en processen in de ready queue plaatsen.

Bij Round Robin worden processorgebonden processen benadeeld door I/O processen omdat deze meer processortijd nodig hebben dan I/O processen en dus veel meer ‘beurten’ nodig hebben om hun taak te volbrengen.

\textbf{Virtual Round Robin} vermijdt bovenstaande oneerlijkheid. Het heeft een FCFS-wachtrij waar de I/O- processen zich in bevinden en wachten op hun I/O-gebeurtenis. Vanzodra deze optreedt worden ze in een aanvullende wachtrij geplaatst. De processen in de aanvullende wachtrij krijgen voorrang op de processen in de ready queue. Deze processen worden niet langer verwerkt dan de tijd die gelijk is aan het algemene tijdquantum min de totale tijd die besteed is aan de uitvoering van het proces sinds het uit de ready queue werd geselecteerd.



\begin{figure}[htp]
    \centering
            \includegraphics[width=4in]{img/virtualroundrobin.png}
        \caption{Virtual Round Robin voorbeeld}
    \label{fig:Virtual Round Robin voorbeeld}
\end{figure}

\textbf{Shortest Process Next}

Dit is een niet-preëmptieve strategie. Het proces met de kleinste verwachte uitvoeringstijd wordt als volgende geselecteerd. Het kortste proces springt dan voorbij langere processen naar het begin van de wachtrij.

\begin{figure}[htp]
    \centering
            \includegraphics[width=4in]{img/spn.png}
        \caption{Shortest Process Next voorbeeld}
    \label{fig:Shortest Process Next voorbeeld}
\end{figure}

Stel dat de geschatte tijd voor een proces niet juist is, dan kan het besturingssysteem het proces onderbreken. De mogelijkheid bestaat dat langere processen aan starvation lijden.

\textbf{Shortest Remaining Time}

Preëmptieve versie van shortest process next strategie. Moet ook de proces tijd schatten en het kortste uitkiezen.

\begin{figure}[htp]
    \centering
            \includegraphics[width=4in]{img/srt.png}
        \caption{Shortest Remaining Time voorbeeld}
    \label{fig:Shortest Remaining Time voorbeeld}
\end{figure}

\textbf{Highest Response Ratio Next}

Deze strategie kiest het volgende proces met de beste ratio.
Ratio = (tijd gespendeerd wachten + verwachte service tijd) / verwachte service tijd

\begin{figure}[htp]
    \centering
            \includegraphics[width=4in]{img/hrrn.png}
        \caption{Highest Response Ratio Next voorbeeld}
    \label{fig:Highest Response Ratio Next voorbeeld}
\end{figure}

\textbf{Feedback Scheduling}

Een nieuw proces komt binnen en krijgt een hoge prioriteit. Elke keer dat hij uitvoeringstijd kreeg, gaat zijn prioriteit een niveau omlaag en komt hij in een wachtrij met een lagere prioriteit.


\begin{figure}[htp]
    \centering
            \includegraphics[width=4in]{img/feedbackscheduling.png}
        \caption{Feedback Scheduling voorbeeld}
    \label{fig:Feedback Scheduling voorbeeld}
\end{figure}

Problemen zijn dat er niet geweten is hoeveel tijd er nog nodig is om het programma uit te voeren en dat grote processen aan starvation kunnen lijden moesten er continu nieuwe processen binnenkomen.

Oplossing hierbij kan zijn om processen de mogelijkheid te geven om zichzelf te promoveren naar een wachtrij met een hogere prioriteit nadat het een tijd niet aan bod zou mogen gekomen zijn.




\textbf{Feedback Performance}

Variaties bestaan, dit is preëmptief op klokinterrupt. Het is vergelijkbaar met Round Robin. Het kan tot starvation leiden!

\begin{figure}[htp]
    \centering
            \includegraphics[width=4in]{img/feedbackperformance.png}
        \caption{Feedback Performance voorbeeld}
    \label{fig:Feedback Performance voorbeeld}
\end{figure}












